%文中に相互参照があるので2回コンパイルしてください.

\documentclass[./main]{subfile}

%\usepackage{amsmath}
\begin{document}
\Chapter{数学記号の歴史(今井)}

\Section{はじめに}
前号の$e^{\pi i}sode$を読んだことがある人は知っていると思いますが,数学科生にしかわからないような難しい数学の話がほとんどでした.この号も前号ほどではないものの,ちょっと難しいなあと思った方もいるのではないでしょうか.まあ東大の数学科生の書く記事なのでそうなるのももっともではありますが,とりあえず五月祭の数学科ブースに来て何気なくこの冊子を手に取った読者の中には,「そんなこと書かれてもわからないよ」という読者もいるでしょう.この記事は,そんな読者も含め文系理系や数学の学習歴に関係なく気軽に楽しめるような記事として書こうと思っています.

さて,前置きはこのくらいにして早速本題に入りますが,皆さんはどのような数学記号を知っているでしょうか.$+$,$-$,$\times$,$\div$,$=$くらいならどんなに数学嫌いの人でもさすがに知っているでしょう.ということで,まずはこれらの記号に根号を表す$\sqrt{\ }$,$\times$と同じく掛け算を表す$\cdot$を加えた7つの記号から始めたいと思います.\ref{A}節は小学校高学年以上,\ref{B}節は中学生以上,\ref{C}節は高校生以上を対象として書いています.

\section{四則演算,冪根,等号}\label{A}
下に書いている通り,現在のように$+$,$-$,$\times$,$\cdot$,$\div$,$=$,$\sqrt{\ }$の記号が使われるようになったのはつい数百年前の16~17世紀,日本でいえば戦国時代か江戸時代初めのころですが,もちろんそれ以前に計算が行われなかったわけではありません.小学校で習うこれら四則演算(足し算,引き算,掛け算,割り算)は今から約4000年前の古代エジプトや古代メソポタミアでも行われていました.それどころか古代メソポタミアについては,2次方程式や3次方程式の解法について書かれた粘土板まであります.しかしこれらの粘土板やエジプトのパピルスには見慣れた数学記号は一切出てきません.ではどのようにして計算を書き記していたのかと思われるでしょう.実は文章で計算をあらわしていたのです.例えば$5\times 3+2=17$は「5の3倍に2を加えると17になる」という具合です(もちろん古代エジプトや古代メソポタミアなので,日本語ではなくヒエログリフで書かれたエジプト語や楔形文字で書かれたシュメール語ではありますが).ところが,単純な計算ならそれで良いですが,複雑な計算を考えるようになるにつれて,文章で表すのは大変になってきます.そこで単語を略したり記号を使ったりするようになりました.$+$,$-$などの登場よりも数百年も前に,インドやギリシャでは単語の頭文字等を演算を表すのに用いていたり,アラビアでは記号を用いて式を書いている記録が残っていますが,その話は割愛します.

\subsection{$+$,$-$}
7つの記号のうち,一番初めに登場したのは$+$と$-$で,確認されている中では1489年に出版されたヴィットマン(J. Widmann, 1460-?)による計算法の本\footnote{``Behende und h\"upsche Rechenung auff allen Kauffmanschafft''『あらゆる商取引の敏捷で上手な計算法』}に登場します.しかし,この本において$+$と$-$は足し算や引き算の記号ではなく,過不足を表す記号として用いられています.例えば,
\begin{quote}
$4+5$と書いて4ツェントネル5ポンド

$3-12$と書いて3ツェントネルに12ポンド不足つまり2ツェントネル88ポンド

(1ツェントネルは100ポンドで,ツェントネル,ポンドは重さの単位)
\end{quote}
のように使ったようです.計算記号として$+$,$-$を使っている本としては1514年に出版されたファンデルフッケ(G. Vander Hoecke, ?-?)の本\footnote{``Een sonderlinghe boeck in dye edel conste Arithmetica''}があります.

$+$,$-$のうち$+$の方は起源がはっきりしています.ラテン語の``et''(接続詞,~と~)の走り書きです.一方,$-$の方については諸説あり,ラテン語の``minus''(より少ない)の省略形\~{m}をさらに省略した形という説もあります.

\subsection{$\sqrt{\ }$}
$+$,$-$の登場から数十年遅れて,根号の記号として$\sqrt{\ }$が登場します.1525年ルドルフ(C. Rudolff, 1499-1545)による代数学教科書\footnote{``Die Co\ss ''}において,平方根を$\sqrt{}$と表しています.その後シュティフェル(M. Stifel, 1486?-1567)によって改訂された第2版(1553)\cite{RudolffStifel}において,立方根,4乗根,5乗根については$\sqrt{}$の後にそれぞれ筆記体のce,zz,\ss のような記号(それぞれ3乗,4乗,5乗を表すcubice, zensizensice, sursolideの略)を書き加えることで表しています.根を表すラテン語の``radix''の頭文字rが由来という説もありますが,由来ははっきりとはわかっていません.上の横線は元々はなく,横線を加えて$\sqrt{\ }$を使ったのは,哲学者としても有名なデカルト\footnote{\label{幾何学}``La G\'eom\'etrie''『幾何学』(1637)}(R. Descartes, 1596-1650)です.

\subsection{$=$}
次に登場したのは$=$です.1557年にイギリスのレコード(R. Recorde, 1512?-1558)が書いた代数学の本\footnote{``The whetstone of witte whiche is the seconde parte of Arithmetike: containyng the 
extraction of rootes: the cossike practise, with the rule of equation: and the woorkes of surde nombers.''}において,``is equalle to''(~は~と等しい)という言葉の繰り返しを避けるために「2本の平行な線は最も等しい2つのものですから\footnote{``bicause noe 2, thynges, can be moare equalle''}」,$=$を使う,としています.

\subsection{$\times$,$\div$,$\cdot$}\label{乗除}
$\times$を初めて掛け算の記号として用いたのは,イギリスのオートレッド(W. Oughtred, 1574-1660)であり,彼の1631年の著書``Clavis mathematicae''『数学の鍵』に出てきます.また,$\div$はスイスのラーン(J. Rahn, 1622-1676)による``Teutsche Algebra''(1659)が初出と言われています.$\times$については由来は不明ですが,$\div$については元々ラテン語の``divisa est''(割られた)の``est''を表す記号で,``divisa $\div$''の``divisa''が省略されたという説があります.

高校数学以降乗算(掛け算)の記号として使われる$\cdot$を初めて使ったのはイギリスのハリオット(T. Harriot, 1560-1621)\footnote{彼の死後に出版された``Analyticae Praxis ad Aequationes Algebraicas Resolvendas''『代数的方程式を解くための解析的方法』(1631)}と言われていますが,広く使われるようになったのは,微積分の発見で有名なライプニッツ(G. W. Leibniz, 1646-1716)がヨハン・ベルヌーイ(Johann Bernoulli, 1667-1748)に送った手紙(1698年7月29日付)に記されてからです.ライプニッツはこの手紙の中で,\\
「$x$と混同しやすいので$\times$は好きではない.私はしばしば単に2つの数量の間に点を書いて$ZC.LM$によって積を表す.」\\
と述べています.

\section{文字式,指数,小数}\label{B}
さて,四則演算や冪根,等号を表す記号ができて,ヨーロッパにおける記号代数学が始まったわけですが,この記号だけでは,当時最先端の数学の数式を書くには多少不便でした.というのも,求めたい未知数を文字で表すことは3世紀のギリシャでも行われていましたが,既知の定数を文字で表すということはしていなかったため,今の記法でいう$x^2+ax+b=0$等のいわゆる一般の2次方程式などを書き表すことはできず,具体的に係数がわかっている方程式しか解くことができなかったのです.

\subsection{既知数を表す文字}
フランスの数学者ヴィエト(F. Vi\`ete, 1540-1603)は,既知数を文字で表した最初の数学者であり,このため,「代数学の父」と呼ばれることもあります.ヴィエトは,既知数を母音($A$,$E$,$I$など),未知数を子音($B$,$D$など)で表しました.彼の死後,1646年に出版された,彼の数学における業績をまとめた``Opera Mathematica''『数学著作集』から引用すると,
\begin{quote}
Si $A\ quad. +B2\ in\ A$, \ae quetur $Z\ plano$. $A+B$ esto $E$. Igitur $E\ quad.$, \ae quabitur $Z\ plano+B\ quad.$

Confectarium.

Itaque, $\sqrt{Z\ plani+B\ quad.}-B$ fit $A$, de qua primum qu\ae rebatur.\smallskip

(日本語訳)

($A$の二乗$+B\cdot 2\cdot A$)が(面積$Z$)に等しいとする.$A+B$を$E$とおく.

すると,$E$の二乗は(面積$Z$+$B$の二乗)に等しいだろう.

結論

これより,初めの方程式から,($\sqrt{\mbox{面積}Z+B\mbox{の二乗}}-B$)が$A$になることがわかる.
\end{quote}
と書いてあります\footnote{面積(planus,元々は「平らな」という意味の形容詞で格変化した形がplano, plani)と書いてあるのが気になるかもしれません.当時の数式は幾何学的な意味を伴っているとみなされていたため,未知数の1乗は長さ,2乗は面積,3乗は体積を表していました.そのため,体積と長さ,面積と体積など,次元が違う者同士の足し算や引き算等は意味がないとみなされ,しないことになっていました.この例の場合,($A$の二乗$+B\cdot 2\cdot A$)は面積を表すので,それと等しい$Z$も面積でなければならなかったのです.}.つまり,既知数を文字で表すことで,2次方程式の解の公式を書き表すことに成功したのです.

\subsection{指数,小数の記法}
現在では指数は$a^3$のように右肩に指数を書くことで表しますが,この記法に落ち着くまでには様々な紆余曲折がありました.前節の引用部分では「2乗された」という意味のラテン語``quadratus''の略$quad.$を使って2乗を表しています.また,\ref{乗除}節に出てきたオートレッドは2乗を$q$,3乗を$c$として,$A^{10}$を$Aqqcc$,つまり$A$の($2+2+3+3$)乗と表しています.これらの記法に代わって,ベルギーのステヴィン(S. Stevin, 1548-1620)は,未知数の冪乗,つまり今でいう$x^1$,$x^2$,$x^3$,$\cdots$ を\textcircled{\scriptsize 1},\textcircled{\scriptsize 2},\textcircled{\scriptsize 3},$\cdots$ と表し,さらに丸の中の数字を分数にすることで$x^{\frac{2}{3}}$なども表せるようにしました.未知数が複数あるときは丸付き数字の前に第2の未知数は$sec$,第3の未知数は$ter$,$\cdots$と書くことで表しました.例えば,現代の記法では
\[
2x^{\frac{2}{3}}y^4+3z-y^3z^{\frac{1}{2}}
\]
と書く多項式を,
\[
2\;\mbox{\textcircled{\scriptsize $\frac{2}{3}$}}\;M\;sec\;\mbox{\textcircled{\scriptsize 4}}+3\;ter\;\mbox{\textcircled{\scriptsize 1}}-sec\;\mbox{\textcircled{\scriptsize 3}}\;M\;ter\;\mbox{\textcircled{\scriptsize $\frac{1}{2}$}}
\]
と表しました.

また,彼はこの指数の記法と同じ記号を用いて,小数を書き表しました\footnote{``De Thiende''『10分の1』(1585)}.例えば,ステヴィンの記法では27.847を
\[
27\mbox{\textcircled{\scriptsize 0}}8\mbox{\textcircled{\scriptsize 1}}4\mbox{\textcircled{\scriptsize 2}}7\mbox{\textcircled{\scriptsize 3}}\quad\mbox{または}\quad 2\overset{0}{7}\overset{1}{8}\overset{2}{4}\overset{3}{7}
\]
と書きます.小数といえば,「この箱の重さは1.5kg」「100m走の世界記録は9.58秒」のように現代ではだれでも当たり前のように使っていますが,実は16世紀以前の人々は小数というものを知らず,整数で表しきれない端数は分数を使って表していました.

指数と小数を現在のような形で書き始めたのは,それぞれ,デカルト\footnotemark[\getrefnumber{幾何学}]とネイピア\footnote{``Rabdologia''『棒計算術』(1617)}(J. Napier, 1550-1617)ですが,デカルトはステヴィンとは違い,指数が正の整数のときにしか使いませんでした.現在のように,指数が分数か整数か,正か負かにかかわらず,右肩に書いて表したのは,万有引力の法則や微積分の発見で有名なニュートン(I. Newton, 1643-1727)がロンドン王立協会初代事務局長のオルデンブルク(H. Oldenburg, 1618-1677)に宛ててた1676年6月13日の手紙の中で,非整数冪に一般化された二項定理の説明をする際に用いたのが最初とされています.文字指数($a^b$のような書き方)もニュートンが導入しました.

\section{微積分}\label{C}
お気づきの読者もいるかもしれませんが,前節に出てきたニュートンと\ref{乗除}節に出てきたライプニッツの両方について「微積分の発見で有名」と書きました.しかし,2人は一緒に研究して微積分を発見したわけではありません.実は彼らはほぼ同時期に別々に微積分を発見したのです.

\subsection{$\dot{y}$}
1665年5月20日,1枚の紙にニュートンは「流率(fluxio\footnote{\label{ラテン語}fluxio, fluens, differentialis, integralisという単語は,当時の科学界の公用語であるラテン語です.ニュートンとライプニッツの論文もラテン語で書かれています.})」を表すためにドットを上に付けた記号を書きました.「流率」とは瞬間的な変化率,つまり今でいう時間$t$で微分した時の導関数のことです.ニュートンは物体の運動から微積分の概念を思いつきました.つまり流率とは速度のことなのです.逆に,速度の関数から物体がどれだけ進んだかを求める積分のことは「流量(fluens\footnotemark[\getrefnumber{ラテン語}])」と表しました.

ところで,この紙は出版はされず,出版物でこの流率の記号が現れるのは,28年後の1693年,イギリスの数学者ウォリス(J. Wallis, 1616-1703)の『代数学』``Algebra''(1685)を改訂してラテン語版\cite{Wallis}を出版する際にニュートンが書き足した部分です.この本において,ニュートンは,$y$の流率(つまり現在の記法で$\frac{dy}{dt}$)は$\dot{y}$,その流率(つまり2階導関数,$\frac{d^2y}{dt^2}$)は$\ddot{y}$,その流率(つまり3階導関数,$\frac{d^3y}{dt^3}$)は$\dot{\ddot{y}}$,その流率(つまり4階導関数,$\frac{d^4y}{dt^4}$)は$\ddot{\ddot{y}}$と書き,
\[
\frac{yy}{b-x},\;\sqrt{aa-xx}\quad(aa,xx,yy\mbox{それぞれ}a^2,x^2,y^2\mbox{を表す})
\]
の流率とそのまた流率はそれぞれ,
\[
\frac{\d{$yy$}}{\dot{b-x}},\;\dot{{\sqrt{\dot{\hspace{-1.zw}aa-xx\hspace{1.zw}}}}\hspace{1.zw}}\mbox{と}\quad \frac{\d{$y$\hspace{0.35zw}}\hspace{-0.7zw}\d{\hspace{0.35zw}$y$}}{\ddot{b-x}},\;\ddot{{\sqrt{\ddot{\hspace{-1.zw}aa-xx\hspace{1.zw}}}}\hspace{1.zw}}
\]
と書きました.この本にはドットを付けて表す流率,つまり時間$t$で微分したものしか載っていませんが,別のところで,$t$以外で微分したもの,例えば現代の記法で$\frac{dy}{dx}$に当たるものとして,$\dot{y}:\dot{x}$と書いています.ここで,$:$は割り算の意味で使われています.

しかし,上記の分数や根号の流率は印刷するのが難しかったため,あまり好まれず広まらなかったようです.$\dot{x}$,$\dot{y}$の形だけは現在でも残っていて,物理学で時間での微分を表すのに使われています.

\subsection{$\frac{dy}{dx}$}
1675年11月11日,ライプニッツの手記において,初めて$dx$,$dy$が$x$,$y$の微分($x$,$y$を動かしたときに,$x$,$y$が動いた長さ)として使われました.ライプニッツの場合,曲線とその接線の関係性を調べる方法として微積分を編み出したため,$dx$や$dy$といった概念が必要となりました.これらを``differentialis\footnotemark[\getrefnumber{ラテン語}]''(微分\footnote{ライプニッツは$x$,$y$の動きが無限小の場合に限らず,$dx$,$dy$をただ単に差として使っていて,直訳だと「差分」の方が近いです.このことは,differentialisという単語が,英語のdifference(差)と似ていることからも推察できます.})と名付けたのも彼です.$dx$,$dy$の$d$は``differentialis''の頭文字です.

ライプニッツは同じ手稿の中で$x$を$y$で微分した導関数を$\frac{dx}{dy}$と表しました.ニュートンのときと同様,これは出版されず,出版物で$dx$,$dy$が初めて登場したのは,1684年に科学雑誌``Acta Eruditorum''\cite{Acta1684}にライプニッツが寄稿した論文です.論文においては,手稿とは異なり,$x$を$y$で微分した導関数は$:$を用いて,$dx:dy$と表し,$\sqrt[b]{X^a}$の微分は$d,\sqrt[b]{}X^a$($d$の直後のカンマに注意),$\frac{1}{X^a}$の微分は$d\frac{1}{X^a}$と表しました.また,1693年の``Acta Eruditorum''\cite{Acta1693}を見ると,2階導関数は$ddx:d\bar{y}^2$と書かれています.$d\bar{y}^2$の上線は,括線とよばれ,括弧と同様の役割を果たしました.この場合,$d\bar{y}^2$は現在の記法では$d(y)^2$ということになりますが,($d\{(y)^2\}$ではなく)$\{d(y)\}^2$の意味で使っていたようです.

ライプニッツの記法は多くのスペースを必要とするものの,どの変数で微分しているかが明確に表せたため,後に多変数関数の微分を考えるようになってから重宝され,今日でも幅広く使用されています.

\subsection{$f'(x)$}
高校数学では,ほとんどの場合導関数を表すために$'$をつけて表しますが,この記法はニュートンのものでもライプニッツのものでもありません.この記法はこの2人の時代より1世紀ほど経った1759年,イタリア生まれで主にフランスで活躍したラグランジュ(J. L. Lagrange, 1736-1813)が,論文集``Miscellanea Taurinensis''の一部として書いた論文に現れます.ラグランジュは,その後``Th\'eorie des fonctions analytiques''『解析関数論』(1797)で微積分学に大きな影響を残しました.ラグランジュの記法では$x$の関数を$fx$と書き,1階導関数を$f'x$,2階導関数を$f''x$,3階導関数を$f'''x$,と書くとし,$y$が$x$の関数のとき,導関数を同様に$y'$,$y''$,$y'''$,$\cdots$と書くとしています.$x$に括弧がついていないことを除けば,現代の記法と同じです.また,ニュートンの記法との違いは,変数$x$を明示することでどの変数で微分したかわかるようになっていることです.ライプニッツの記法に比べスペースも少なく済むため,今日まで生き残っているのだと思われます.

\subsection{$\frac{\partial f}{\partial x}$}
多変数関数の微分を考えるようになると,数学者たちは偏微分と全微分を書き分ける必要が出てきました.$f$の$x$による偏導関数を表す記法のいくつか例を挙げると,
\begin{quote}
オイラー(L. Euler, 1707-1783)\quad $\left(\frac{df}{dx}\right)$(1755,全微分は括弧なし)

カーステン(W. J. G. Karsten, 1732-1787)\quad $frac{\overset{x}{d}f}{dx}$(1760)

フォンテーヌ(A. Fontaine, 1704-1771)\quad $\frac{df}{dx}$(1764,全微分は$\frac{1}{dx}\cdot df$)
\end{quote}
などです.参考文献の\cite{Cajori1923}を見ると他にも色々な記法が使われていたことがわかります.現代の記法である$\frac{\partial f}{\partial x}$はルジャンドル(A. M. Legendre, 1752-1833)が1786年の論文で導入します.しかし,しばらくの間,多くの数学者がオイラーやフォンテーヌの記法を使い続けます.ルジャンドル自身,後の論文では全微分と偏微分の記法を区別しなかったりしたようです.さらに,当時数学の別の分野(数値解析の有限差分法)において$d$を使っていたため,全微分を$\partial$を使って表したりする人もいて,記法は定まることなくさらに数十年が経過します.1841年,ヤコビ(C. G. J. Jacobi, 1804-1851)が論文``De determinalibus functionalibus''で再び$d$と$\partial$による全微分と偏微分の書き分けを導入し,その後50年ほど経って,19世紀の終わりごろに一般的な記法とみなされるようになりました.

\subsection{$\int$}
$\int$という記号は,1675年10月29日のライプニッツの手稿に初めて現れます.このとき,ライプニッツは``$\int l$ pro omn. $l$''と書いて,「全てのlの和」という意味で使っていました.$\int$という記号も,和を意味するラテン語の``summa''(当時は語末以外の小文字のsはfの横棒を消したような形の文字(以後「長いs」と呼びます)で書いていました.)の頭文字の「長いs」のイタリック体です.``integralis\footnotemark[\getrefnumber{ラテン語}]''(積分)という用語を考えたベルヌーイは,イタリック体のIを使ってはどうかと提案しましたが,最終的にはライプニッツに敬意を表して$\int$を使うことにしたそうです.出版物で$\int$が初めて出てきたのは1686年で,これも``Acta Eruditorum''の論文が初出ですが,このときはもっと「長いs」に近い形だったようです.

1675年10月29日の手稿には$\int x^2=\frac{x^3}{3}$と書いてあります.現代の記法と比べると積分の最後に$dx$がないことがわかります.一方,1686年の出版された論文では,ライプニッツは最後に$dx$を書いています.積分を微分の逆としてみるなら,確かに$dx$は不要かもしれません.実際,$dx$を書かず,$\int_x$と書いた数学者もいたようです.

1819-20年には,フーリエ(J. Fourier, 1768-1830)が定積分の記法$\int_a^b$を考案しました.偏微分の記号とは全く違い,こちらは概ね好評で,すぐに自分の論文に取り入れた数学者もいました.


\subsection{ニュートンの積分記号}
微分に関してはニュートンの記法もライプニッツの記法も現在まで引き継がれていますが,積分に関してはどうでしょう.$\int$とその派生形である$\oint$以外を目にしたことがある人はほとんどいないのではないでしょうか.ニュートンも当然,流量(積分)の記号は考案して使っていました.

ニュートンの記法は2つありました.まず1つ目は文字の上に縦棒を書く方法\footnote{``Quadratura curvarum''(1704)}です.例えば,$x$の流量は$x^{{}^{\hspace{-0.4zw}|}}$,その流量は$x^{{}^{\hspace{-0.5zw}||}}$といった具合です.

2つ目は四角で囲む方法です.例えば,現代の記法で書くと$\int \frac{a^2\;dx}{64x}$という式を,ニュートンは``De Analysi per \ae quationes numero terminorum infinitas''(1669)において\fbox{$\frac{aa\cdot dx}{64x}$}と書いています.

ニュートンの記法が普及しなかったのは,1つ目に関しては$x'$と紛らわしく,2つ目に関しては,印刷するのが大変だったためということのようです.

\Section{結び}
もちろんこれ以外にも数学記号はたくさんあるのですが,それを思いつくままに片っ端から由来を調べていたのではきりがないので,この記事はこのあたりで終わりにしたいと思います.この記事を書くにあたって私も利用しましたが,数百年前の数学書は著作権も当然切れていて,ネットで全文公開されているものも少なくありません.ほとんどがラテン語で書かれているので大多数の人は読めないとは思いますが,数式や図を見てみても,現代の数学書とはだいぶ雰囲気が違います.暇な時にでも見てみるといいかもしれません.ここまで読んでくださりありがとうございました.

\begin{thebibliography}{99}
\bibitem{関数}岡本久,長岡亮介 (2014). {\it 関数とは何か\quad 近代数学史からのアプローチ.} 東京:近代科学社
\bibitem{Cajori1925}Cajori, F.著, 小倉金之助補訳 (1997). {\it カジョリ初等数学史.} 東京:共立出版
\bibitem{数学史}中村滋 (2015). {\it 数学史の小窓.} 東京:日本評論社
\bibitem{Cajori1923}Cajori, F. (1923). The History of Notations of the Calculus. {\it Annals of Mathematics, 25}(1), second series, 1-46. doi:10.2307/1967725
\bibitem{Gerhardt}Gerhardt, C. I. (1855). {\it Leibnizens mathematische Schriften.} Halle: Verlag von A. Asher \& Comp.
\bibitem{Acta1684}Grosse II, J., Gleiditsch II, J. F., et al. (1684). {\it Acta eruditorum: anno MDCLXXXIV publicata.} Leipzig
\bibitem{Acta1693}Grosse II, J., Gleiditsch II, J. F., et al. (1693). {\it Acta eruditorum: anno MDCXCIII publicata.} Leipzig
\bibitem{Heeffer}Heeffer, A. (2008). The Emergence of Symbolic Algebra as a Shift in Predominant Models. {\it Foundations of Science, 13}(2), 149-161. doi:10.1007/s10699-008-9124-0 
\bibitem{jeff560}Miller, J. (n.d.). Earliest Uses of Symbols of Operation. Retrieved April 17, 2018, from http://jeff560.tripod.com/operation.html 
\bibitem{Newton}Newton, I. (2008). {\it The Mathematical Papers of Isaac Newton.} Cambridge: Cambridge University Press
\bibitem{RudolffStifel}Rudolff, C., Stifel, M. (1553). {\it Die Coss Christoffs Rudolffs mit schönen Exempeln der Coss / durch Michael Stifel gebessert und sehr gemehrt.} K\"onigsperg: gedr\"uckt durch Alexandrum Lutomyslensem
\bibitem{Wallis}Wallis, J. (1693). {\it De algebra tractatus; historicus \& practicus.} Oxford: Sheldonian Theatre
\end{thebibliography}

\end{document}
