\documentclass[./main]{subfiles}
%\documentclass[a4paper]{jsarticle}
%\usepackage{amsthm,amsmath,txfonts,pxfonts,mathrsfs}
\theoremstyle{break}
\newtheorem*{defi}{定義}
\newtheorem*{thm}{定理}
\newtheorem*{prb}{問題}
\begin{document}
\Chapter{オイラーの無限級数論(植田)}
\setcounter{section}{-1}
\section{注意}
このコラムを読むにあたって、高校数学レベルの知識を仮定しています。また、オイラーの時代に習った古い記法を採用しました。
\section{オイラーって?}
レオンハルト・オイラーは、18世紀において最も偉大だとされる数学者です。彼は極めて多くの論文を残し、その影響範囲は当時最先端の数学の全分野をカバーしているといっても過言ではありません。

彼の時代、数学はまだ厳密な証明を重視しておらず、直観的な無限大や無限小の取り扱いが許されていました。日本の高校数学で取り扱われている直観的な考え方によって、この時代の数学は理解できる、といってもよいでしょう。高校数学には存在しない「数ではないもの」や「図形ではないもの」を取り扱うことができるようになるには、次の世紀を待たねばなりませんでした。

そして、彼の時代において、特定の無限級数の値を正確に求める、ということは、数学の主題の一つでもありました。彼の名を当時の数学界に知らしめることとなった最初の業績も、次の級数の値を正確に求めるという、当時ヨーロッパ中の数学者の挑戦をはねのけ続けていた難問、いわゆるバーゼル問題を解決したということです。

\[1+\frac{1}{2^2}+\frac{1}{3^2}+\frac{1}{4^2}+\frac{1}{5^2}+\cdots\]

そして、このコラムのテーマは、これに類似する無限級数の値のオイラーらしい求め方を知り、自分で値を求められるようになることです。証明は現代のように厳密ではありませんが、結果はすべて正しいものです。
\section{バーゼル問題}
オイラーは上の級数の値を求めるために、関数$\displaystyle\frac{e^x-e^{-x}}{2}$を次の二つの方法で表しました。

\[\frac{e^x-e^{-x}}{2}=x+\frac{x^3}{3!}+\frac{x^5}{5!}+\frac{x^7}{7!}+\frac{x^9}{9!}+\cdots=x\left(  1+\frac{x^2}{\pi^2}\right)\left(  1+\frac{x^2}{2^2\pi^2}\right)\left(  1+\frac{x^2}{3^2\pi^2}\right)\cdots\]

もちろんこれを求める方法もオイラーは記しましたが、それをここに書くにはスペースが足りないので、代わりに

\[\sin x=x-\frac{x^3}{3!}+\frac{x^5}{5!}-\frac{x^7}{7!}+\frac{x^9}{9!}-\cdots=x\left(  1-\frac{x^2}{\pi^2}\right)\left(  1-\frac{x^2}{2^2\pi^2}\right)\left(  1-\frac{x^2}{3^2\pi^2}\right)\cdots\]

を使うことにします。右側は$\sin x$の零点が$x=0,\pm \pi,\pm 2\pi,\pm 3\pi,\cdots$であるので、展開した時xの係数が1になるように因数分解した(いつでもできるわけではない)ものです。

左側の説明には、高校数学の範囲で証明可能な次の定理があれば十分です。

\begin{thm}[テイラーの定理]
 

0に十分近いxにおいて、

\[f(x)=\sum_{n=0}^\infty \frac{x^n}{n!}\frac{d^nf}{dx^n}(0)=f(0)+f'(0)x+\frac{f''(0)}{2!}x^2+\frac{f'''(0)}{3!}x^3+\cdots\]

が成り立つ。
\end{thm}
この定理も、証明を書くのにスペースが足りませんので、知りたい人はご自分でお調べください。

$f(x)=\sin x$の場合は$(\sin x)'=\cos x,(\sin x)''=-\sin x,\sin 0=0,\cos 0=1$より、

\[\sin x=x-\frac{x^3}{3!}+\frac{x^5}{5!}-\frac{x^7}{7!}+\frac{x^9}{9!}-\cdots\]

が言えます。

ここで、$x^2=-\pi^2 z$とおくと、
\[1+\frac{\pi^2}{6}z+\frac{\pi^4}{120}z^2+\frac{\pi^6}{5040}z^3+\frac{\pi^8}{362880}z^4+\cdots=\left(  1+z\right)\left(  1+\frac{z}{2^2}\right)\left(  1+\frac{z}{3^2}\right)\cdots\]

となります。右辺を展開して$z$の係数を比較すると、

\[1+\frac{1}{2^2}+\frac{1}{3^2}+\frac{1}{4^2}+\cdots=\frac{\pi^2}{6}\]

となり、最初の級数の解が得られました。

以上が、オイラーの導き出した答えです。

\begin{prb}
テイラーの定理について、図書館やインターネットなどで証明を探し、目を通しなさい。
\end{prb}

\section{べき指数がより大きい級数}
オイラーは上の級数の値を求めただけで満足はしませんでした。続いて、

\[1+\frac{1}{2^{2n}}+\frac{1}{3^{2n}}+\frac{1}{4^{2n}}+\cdots=\frac{\pi^2}{6}\]

の値を求める方法を発見したのです。

一般に、

\[(1+az)(1+bz)(1+cz)\cdots=1+Az+Bz+Cz+\cdots\]

とすると、$A=a+b+c+\cdots,B=ab+ac+bc+\cdots,C=abc+\cdots$といったように、$A,B,C,\cdots$は解と係数の関係によって、基本対称式になります。それは無限変数であっても例外ではありません。

ここで、$P_n=a^n+b^n+c^n+\cdots$は対称式であり、

\begin{eqnarray*}
P_1&=&A\\
P_2&=&AP_1-2B\\
P_3&=&AP_2-BP_1+3C\\
P_4&=&AP_3-BP_2+3P_1-4D\\
&\vdots&
\end{eqnarray*}

といった形で、基本対称式で表せます。

これを逐次計算していくことで、$P_n$の値を具体的に求めることが可能になります。

例えば、

\[1+\frac{1}{2^4}+\frac{1}{3^4}+\frac{1}{4^4}+\cdots=A^2-2B=\left( \frac{\pi^2}{6} \right)^2-2\cdot\frac{\pi^4}{120}=\frac{\pi^4}{36}-\frac{\pi^4}{60}=\frac{5-3}{180}\pi^4=\frac{\pi^2}{90}\]

\begin{prb}
これに習って、$n=3,4,5,\cdots$の値を求めてみてください。

計算を間違えなければ、$\displaystyle \frac{\pi^6}{945},\frac{\pi^8}{9450},\frac{\pi^{10}}{93555},\cdots$という値が得られるはずです。

\end{prb}
\section{パラメータを持つ級数}
上のような級数の求め方は、同じ関数を表す級数と無限積があれば、いつでも実行できます。

事実、オイラーはより多くの無限級数の値を与える、ある関数の級数表示と無限積表示を示しています。

$\sin x$の級数表示を求めたときと同様にして、

\[\cos x=1-\frac{x^2}{2!}+\frac{x^4}{4!}-\frac{x^6}{6!}+\cdots\]

がわかります。これを用いて、

\[\cos(x+y)=\cos x\cos y-\sin x \sin y=\cos y -x\sin y -\frac{x^2}{2!}\cos y+\frac{x^3}{3!}\sin y\cdots\]

\[\frac{\cos (x+y)}{\cos x}=1 -x\tan y- \frac{x^2}{2!}+\frac{x^3}{3!}\tan y\cdots\]

ここで、$\cos (x+y)=0$となるのは$x+y=\pm \left(n+\frac{1}{2}\right)\pi$の時でしたから、

\[\frac{\cos (x+y)}{\cos x}=1 -x\tan y- \frac{x^2}{2!}+\frac{x^3}{3!}\tan y\cdots=\left( 1-\frac{x}{\frac{1}{2}\pi -y} \right)\left( 1+\frac{x}{\frac{1}{2}\pi+y} \right)\left( 1-\frac{x}{\frac{3}{2}\pi-y} \right)\cdots\]

ここで、$x=\frac{t}{2n}\pi ,y=\frac{1}{2}\pi \frac{m}{n},z=\tan y$とすると、

\[1 -\frac{t}{2n}z\pi - \frac{t^2}{2!\cdot (2n)^2}\pi^2+\frac{t^3}{3!(2n)^3}z\pi^3\cdots=\left( 1-\frac{t}{n-m} \right)\left( 1+\frac{t}{n+m} \right)\left( 1-\frac{t}{3n-m} \right)\cdots\]

この式に対して、前述の基本対称式に当てはめることで

\[\frac{1}{(n-m)^k}+\frac{1}{(-(n+m))^k}+\frac{1}{(3n-m)^k}+\cdots\]

という形で表される級数の値を求めることが可能になります。例えば、$n=2,m=1,k$を奇数に限定すると、

\[1-\frac{1}{3^k}+\frac{1}{5^k}-\frac{1}{7^k}+\frac{1}{9^k}\cdots\]

となります。このとき$y=\frac{\pi}{4},z=1$なので、$k=1$のとき、

\[1-\frac{1}{3}+\frac{1}{5}-\frac{1}{7}+\frac{1}{9}\cdots=\frac{z\pi}{2n}=\frac{\pi}{4}\]

となります。

蛇足ではありますが、これらの級数の族から導かれる特殊な形の級数として、バーゼル問題の級数に定数項を付け加えた級数が存在します。

$k=1$としたとき、

\[\frac{1}{n-m}-\frac{1}{n+m}+\frac{1}{3n-m}-\cdots=\frac{z\pi}{2n}\]

となりますが、この級数を2項ずつまとめると、

\[\frac{2m}{n^2-m^2}+\frac{2m}{9n^2-m^2}+\frac{2m}{25n^2-m^2}\cdots=\frac{z\pi}{2n}\]

となり、整理すると、$m=np$として、

\[\frac{1}{1-p^2}+\frac{1}{9-p^2}+\frac{1}{25-p^2}\cdots=\frac{z\pi}{4p}\]

また、$l=n-m$とすれば、

\[\frac{1}{l}-\frac{1}{2n-l}+\frac{1}{2n+l}-\cdots=\frac{z\pi}{2n}\]

となり、最初の項だけを右辺に移項し、$l=nq$として整理すると

\[\frac{1}{4-q^2}+\frac{1}{16-q^2}+\frac{1}{36-q^2}\cdots=\frac{1}{2q^2}-\frac{z\pi}{4q}\]

となる。これらの級数の和や差をとることによって、

\[\frac{1}{1-p^2}+\frac{1}{4-p^2}+\frac{1}{9-p^2}+\cdots\]

\[\frac{1}{1-p^2}-\frac{1}{4-p^2}+\frac{1}{9-p^2}-\cdots\]

などといった級数の値を求めることができます。しかし、このときpは0以上1以下でなくてはならず、分母の中に分数が入って見栄えがあまりよくありません。

そこで、テイラーの定理によって

\[e^x=1+x+\frac{x^2}{2!}+\frac{x^3}{3!}+\cdots\]

となり、係数を見比べることで$e^{ix}=\cos x+i\sin x$がわかります。これを用いると、

\[\frac{1}{1+b}-\frac{1}{4+b}+\frac{1}{9+b}-\frac{1}{16+b}+\cdots=\frac{1}{2b}-\frac{\pi}{\sqrt{b}(e^{\pi\sqrt{b}}-e^{-\pi\sqrt{b}})}\]

\[\frac{1}{1+b}+\frac{1}{4+b}+\frac{1}{9+b}+\frac{1}{16+b}+\cdots=\frac{\pi(e^{\pi\sqrt{b}}+e^{-\pi\sqrt{b}})}{2\sqrt{b}(e^{\pi\sqrt{b}}-e^{-\pi\sqrt{b}})}-\frac{1}{2b}\]

がわかります。このときは任意のbに対して正確な値を求めることができます。

\begin{prb}
$z=\tan\frac{m\pi}{2n}$の値が具体的にわかるようなn,m(ただしnは12以下、$n-m,n+m$がともに自然数となるものとする)について、級数にn,mを代入して書き出し、いくつかのkについて具体的な値を計算しなさい。

例えば、$(n,m)=(\frac{3}{2},\frac{1}{2})$は$z=\sqrt{3},n-m=1,n+m=2$となり、条件を満たしています。
\end{prb}
\section{無限積の展開}
オイラーは他にもいくつか三角関数の無限積表示と級数表示を与えていますが、得られる級数は上で示したもの以外にはありませんでした。

ですが、オイラーは更に多くの種類の無限級数の値を求める方法を見つけています。

\[\frac{1}{2^s}\left( 1+\frac{1}{2^s}+\frac{1}{3^s}+\frac{1}{4^s}+\cdots \right)=\frac{1}{2^s}+\frac{1}{4^s}+\frac{1}{6^s}+\frac{1}{8^s}+\cdots\]

の右辺は新たな級数であり、右辺の括弧の中の級数の部分級数でもあります。よって、

\[1+\frac{1}{3^s}+\frac{1}{5^s}+\frac{1}{7^s}+\cdots=\left( 1-\frac{1}{2^s} \right)\left( 1+\frac{1}{2^s}+\frac{1}{3^s}+\cdots \right)\]

となります。この左辺もそれ自身新たな級数として値を求めるに値すると思いますが、ここからさらに、

\[\frac{1}{3^s}\left( 1+\frac{1}{3^s}+\frac{1}{5^s}+\frac{1}{7^s}+\cdots \right)=\frac{1}{3^s}+\frac{1}{9^s}+\frac{1}{15^s}+\frac{1}{21^s}+\cdots\]

\[1+\frac{1}{5^s}+\frac{1}{7^s}+\frac{1}{11^s}+\cdots=\left( 1-\frac{1}{3^s} \right)\left( 1+\frac{1}{3^s}+\frac{1}{5^s}+\cdots \right)=\left( 1-\frac{1}{2^s} \right)\left( 1-\frac{1}{3^s} \right)\left( 1+\frac{1}{2^s}+\frac{1}{3^s}+\cdots \right)\]

となります。

同様の操作は2以上の任意の自然数に対して同様に行うことができますので、1個あるいは複数個の特定の約数を分母が持つ項だけの和、もしくは持たない項だけの部分級数の値はこの方法で求めることができます。ただし、指定した約数の中に1以外の公約数を持つものがあれば、調整が必要です。

また、この方法で、同じ条件の項を取り除くのではなく符号を反転させた級数、例えば

\[1-\frac{1}{2^s}+\frac{1}{3^s}-\frac{1}{4^s}+\cdots\]

の値を求めることも、

\[1-\frac{1}{2^s}+\frac{1}{3^s}-\frac{1}{4^s}+\cdots= \left(1+\frac{1}{2^s}+\frac{1}{3^s}+\frac{1}{4^s}\cdots \right)-2\left(\frac{1}{2^s}+\frac{1}{4^s}+\frac{1}{6^s}+\frac{1}{8^s}+\cdots \right)\]

\[\left( 1+\frac{1}{2^s}+\frac{1}{3^s}+\frac{1}{4^s}\cdots \right)\left( 1-\frac{1}{2^s} \right)\left( 1-\frac{1}{3^s} \right)\left( 1-\frac{1}{5^s} \right)\cdots=1\]

となるので可能です。しかし、この場合は互いに素な数を約数に持つ場合も、たとえば2と3を約数に持つ場合、6を約数に持つ項が二重に引かれてしまうので、6を約数に持つ項の部分和を足さなくてはいけません。

4章で示したパラメータを持つ級数の場合も、一部のパラメータに対してはまったく同様に部分和や符号を反転させた和を求めることが可能です。

\[\frac{1}{3^s}\left(1-\frac{1}{3^s}+\frac{1}{5^s}-\frac{1}{7^s}\cdots\right)=\frac{1}{3^s}-\frac{1}{9^s}+\frac{1}{15^s}\cdots\]

\[\left( 1+\frac{1}{3^s} \right)\left( 1-\frac{1}{3^s}+\frac{1}{5^s}-\frac{1}{7^s}\cdots \right)=1+\frac{1}{5^s}-\frac{1}{7^s}-\frac{1}{11^s}\cdots\]

などとして、これも任意の奇数に対して同様の操作を行うことができます。このような操作が実行可能であるためには、パラメータnが12を割り切る必要があります。

これらの級数に対しすべての素数にわたって同じ操作を行うことで、

\[1+\frac{1}{2^s}+\frac{1}{3^s}+\frac{1}{4^s}\cdots =\frac{1}{\left( 1-\frac{1}{2^s} \right)\left( 1-\frac{1}{3^s} \right)\left( 1-\frac{1}{5^s} \right)\cdots}\]

\[1-\frac{1}{3^s}+\frac{1}{5^s}-\frac{1}{7^s}\cdots =\frac{1}{\left( 1+\frac{1}{3^s} \right)\left( 1-\frac{1}{5^s} \right)\left( 1+\frac{1}{7^s} \right)\cdots}\]

などという級数が得られます。2つ目の級数においては、分母の素数が4で割って3余るとき符号が+,1余るとき符号が-となります。

無限積を展開するとどのような級数が得られるか、ここで確認しておきます。

\[(1+az)(1+bz)(1+cz)\cdots=1+Az+Bz+Cz+\cdots\]

としたときの$A,B,C,$といった個別の値はすでに確認しました。z=1とおくと、

\[(1+a)(1+b)(1+c)\cdots=1+A+B+C+\cdots\]

となり、$A,B,C,\cdots$は相異なる1個、2個、3個の$a,b,c,\cdots$の積の和だったことを考えると、右辺は相異なる$a,b,c,\cdots$の積として表される任意の数の和となります。

では、$z=-1$とした場合はどうでしょうか。

\[(1-a)(1-b)(1-c)\cdots=1-A+B-C+\cdots\]

となりますが、符号が正になっているのは相異なる偶数個の$a,b,c,\cdots$の積の和となるもの、負になっているのは相異なる奇数個の積の和となっているものです。

これより、右辺は、相異なる偶数個の$a,b,c,$の積として表される任意の数の和から相異なる奇数個の$a,b,c$の積として表される任意の数の和を引いたものとなります。

いったん元の形から離れて、$(1+a+a^2)(1+b+b^2)(1+c+c^2)\cdots=$とした場合はどうなるでしょう。この場合は、$a,b,c,\cdots$の積であって、同じ$a,b,c,\cdots$の元が高々2個までで表されるような任意の数の和です。

より一般に、

\[(1+\alpha_1a+\alpha_2a^2+\cdots)(1+\beta_1b+\beta_2b^2+\cdots)(1+\gamma_1c+\gamma_2c^2+\cdots)\cdots\]

とすると、各項は$a,b,c,\cdots$の積として表される任意の数に、$a,b,c,\cdots$の次数$k_a,k_b,k_c,\cdots$に応じて、$\alpha_{k_a}\beta_{k_b}\gamma_{k_c}\cdots$を掛けたものです。

ここで、任意のiに対して$1=\alpha_i=\beta_i=\cdots$とすると、$1+x+x^2+x^3\cdots=\frac{1}{1-x}$だったので、左辺の無限積は

\[\frac{1}{(1-a)(1-b)(1-c)\cdots}\]

に等しくなります。$a,b,c,\cdots$を素数の逆数のs乗とすれば、任意の自然数はただ一つの素因数分解を持つので、

\[1+\frac{1}{2^s}+\frac{1}{3^s}+\frac{1}{4^s}\cdots =\frac{1}{\left( 1-\frac{1}{2^s} \right)\left( 1-\frac{1}{3^s} \right)\left( 1-\frac{1}{5^s} \right)\cdots}\]

というすでに示した結果が得られます。

また、$\frac{1}{1+x}=1-x+x^2-x^3+\cdots$だったので、

\[1-\frac{1}{3^s}+\frac{1}{5^s}-\frac{1}{7^s}\cdots =\frac{1}{\left( 1+\frac{1}{3^s} \right)\left( 1-\frac{1}{5^s} \right)\left( 1+\frac{1}{7^s} \right)\cdots}\]

の右辺は、分母の素因数の中に4で割って3余る数で、次数が奇数個のものが奇数個あるとき、その項の符号は-,そうでないときその項の符号は+ということです。

4で割って3余る素因数で次数が奇数個のものが奇数個あるということと4で割って3余る素因数が重複も含めて奇数個あるということは同じことであり、また、4で割って3余る奇数個の数といくつかの4で割って1余る数の積は4で割って3余るので、4で割って3余る数を分母に持つ項は-,4で割って1余る数を分母に持つ項は+となり、左辺と一致します。

ここまで素数のみについて述べましたが、$\sin x$の無限積表示と$e^{ix}=\cos x+i\sin x$だけを用いて、自然数全体、あるいは偶数のみ、奇数のみ、また6と素な自然数といった集合上にわたる様々な無限積を求めることが可能です。

それらは$\left( 1-\frac{z^k}{n^k} \right)$を因数分解することによって得られますが、kが偶数の場合のほかkが3の場合にも値を求めることが可能です。この際n項目の分子はnを特定の条件に従って自然数の積で表す方法の数となります。

ここで、無限積の中の有限個の項の符号を反転させたとき、級数は無限個の項が反転し、かつ、元の無限積の値がわかっていれば、その値に有理数の掛け算を有限回行うだけで、無限個の級数が反転した値を求めることができます。

しかし、無限積の中のすべての項の符号を反転させた無限積、ひいてはそれを展開して得られる級数の値を求める方法も存在します。

\[\left( 1\mp \frac{1}{2^s} \right)\left( 1\mp \frac{1}{3^s} \right)\left( 1\mp\frac{1}{5^s} \right)\cdots=\frac{\left( 1-\frac{1}{2^{2s}} \right)\left( 1-\frac{1}{3^{2s}} \right)\left( 1-\frac{1}{5^{2s}} \right)\cdots}{\left( 1\pm \frac{1}{2^s} \right)\left( 1\pm \frac{1}{3^s} \right)\left( 1\pm\frac{1}{5^s} \right)\cdots}\]

となります。3章の方法によって右辺の分子の無限積はいつでも求めることが可能なので、右辺の分母の値がわかればその無限積のすべての項の符号を反転させた形である左辺の無限積の値もわかります。

また、興味深い点として、

\[\left( 1\mp \frac{1}{2^s} \right)\left( 1\mp \frac{1}{3^s} \right)\left( 1\mp\frac{1}{5^s} \right)\cdots\]

を級数展開した形は

\[\frac{1}{\left( 1\pm \frac{1}{2^s} \right)\left( 1\pm \frac{1}{3^s} \right)\left( 1\pm\frac{1}{5^s} \right)\cdots}\]

を級数展開した形の部分級数である、という事実があります。このことから、双方の値を求めて級数展開し、下の無限級数から上の無限級数を引くことで新たな部分和の形が求められます。

さらに、二つの無限積の次数sが等しく、符号だけが異なる場合は、片方をもう片方で割ると、各素数nの項が$\displaystyle\frac{n^s-1}{n^s+1},\frac{n^s+1}{n^s-1}$のどちらかとなりますが、

\[\frac{n^s-1}{n^s+1}=1-\frac{2}{n^s}+\frac{2}{n^{2s}}-\frac{2}{n^{3s}}+\cdots,\frac{n^s+1}{n^s-1}=1+\frac{2}{n^s}+\frac{2}{n^{2s}}+\frac{2}{n^{3s}}+\cdots\]

となり、級数展開した時各項の分子は分母を$n^s$として、nの素因数の個数を$\omega(n)$とすると、$2^{\omega(n)}$と表すことができます。このように、一部の無限積においては級数展開の分子の形を純粋に自然数nの整数論的性質によって記述することが可能です。

自然数から自然数への関数fがそのような性質を持つためには、自然数nの素因数を$p_1,\cdots,p_m,$指数を$k_1,\cdots,k_m$として、$\displaystyle\frac{f(n)}{n^s}=\frac{f(p_1^{k_1})}{p_1^s}\cdots\frac{f(p_m^{k_m})}{p_m^s}$となる必要があるので、$f(n)=f(p_1^{k_1})\cdots f(p_m^{k_m})$という性質を持たなくてはいけません。

そのような性質を持つ重要な関数として、nの約数の数を表す$d(n)$や、n以下でnと素な自然数を表す$\phi(n),$またnの素因数分解の指数の積を表す$\lambda(n),$最後に任意の自然数kに対する$k^{\omega(n)}$があります。また、これら同士の積や、nの代わりに$n^k$を代入したものが条件を満たすことは容易に証明できます。

素数の冪に対して値を求めてみると、

\[d(p^m)=m+1,\phi(p^m)=p^m-p^{m-1},\lambda(p^m)=m,k^{\omega(p^m)}=k\]

となります。

また、dや$\phi$を特別な場合として含み、上の性質を満たす関数として、nのすべての約数のk乗の和をあらわす$\sigma_k(m)$や、n以下の自然数k個の(順番を入れ替えたものは区別する)組であって全体でnと公約数を持たないものの数をあらわす$J_k(n)$というものも存在します。このとき、$\sigma_0(n)=d(n),J_1(n)=\phi(n)$です。

素数の冪に対しては、

\[\sigma_k(p^m)=1+p^k+\cdots+p^{km},J_k(p^m)=p^{mk}-p^{(m-1)k}\]

となります。これらを用いて実際に無限積表示を求めてみると、例えば

\[1+\frac{2^{\omega(p)}\lambda(p)}{p^s}+\frac{2^{\omega(p^2)}\lambda(p^2)}{p^{2s}}+\frac{2^{\omega(p^3)}\lambda(p^3)}{p^{3s}}\cdots=1+\frac{2}{p^s}+\frac{4}{p^{2s}}+\frac{6}{p^{3s}}\cdots=1+\frac{2}{p^s}\frac{1}{\left( 1-\frac{1}{p^s} \right)^2}=\frac{\left( 1+\frac{1}{p^{2s}} \right)}{\left( 1-\frac{1}{p^s} \right)^2}\]

\[1+\frac{J_k(p)}{p^s}+\frac{J_k(p^2)}{p^{2s}}+\frac{J_k(p^3)}{p^{3s}}+\cdots=1+\frac{1}{p^{s-k}}-\frac{1}{p^s}+\frac{1}{p^{2s-2k}}-\frac{1}{p^{2s-k}}+\cdots=\frac{1-\frac{1}{p^s}}{1-\frac{1}{p^{s-k}}}\]

などがわかります。

\begin{prb}
この章で示した様々な事実をもとに、自分しか値を知らない級数を見つけなさい。
\end{prb}
\end{document}
