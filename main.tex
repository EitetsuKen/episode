\documentclass[notombow,episode,openany,dvipdfmx]{kyouritu}
\usepackage{graphicx,color}
\usepackage{tikz}
\usepackage{makeidx,multicol}
\usepackage{amsmath,amssymb,amsthm}
\usepackage{standalone}
\usepackage{tocloft}
\usepackage{aliascnt}
\usepackage{ascmac}
\usepackage{subfiles}

% ueda.texから移動:
\usepackage{txfonts,pxfonts,mathrsfs}

% standard.sty から移動:
\usepackage{bm}
\usepackage[top=25truemm,bottom=20truemm,left=20truemm,right=20truemm]{geometry}
\usepackage{setspace} % setspaceパッケージのインクルード\usepackage{wrapfig}
\usepackage{multicol}
\usepackage{ulem}
\usepackage{url}
\usepackage{array,arydshln}

\usepackage{hyperref}\usepackage{pxjahyper}
%\usepackage{natbib} 
%\usepackage{standard}
%プリアンブル
\topmargin -0.5in
\headheight 0.2in
\headsep 0.3in  
\evensidemargin -0.03in
\oddsidemargin -0.4in
%\textwidth 5.6in
%\textheight 8.4in\renewcommand{\thetable}{%
%\arabic{table}}

%圏点
\makeatletter
\def\kenten#1{%
\ifvmode\leavevmode\else\hskip\kanjiskip\fi
\setbox1=\hbox to \z@{・\hss}%
\ht1=.63zw
\@kenten#1\end}
\def\@kenten#1{%
\ifx#1\end \let\next=\relax \else
\raise.63zw\copy1\nobreak #1\hskip\kanjiskip\relax
\let\next=\@kenten
\fi\next}
\makeatother

%Section等先頭を大文字にすると番号付けしない.
\newcommand{\Chapter}[1]{\chapter*{{\Huge #1}}
\markboth{#1}{#1}
\addcontentsline{toc}{chapter}{#1}
\stepcounter{chapter}}
\newcommand{\Section}[1]{\section*{{\huge #1}}
\addcontentsline{toc}{section}{#1}}
\newcommand{\Subsection}[1]{\subsection*{\underline{#1}}}
\newcommand{\Subsubsection}[1]{\subsubsection*{#1}}
\setcounter{tocdepth}{0} %Chapterのみ表示する
\renewcommand*{\sectionmark}[1]{}
\renewcommand{\thesection}{\arabic{section}}
\renewcommand{\thesubsection}{\thesection.\arabic{subsection}}
\renewcommand{\theequation}{\arabic{equation}}
\renewcommand*{\KBsectionfont}[1]{\normalfont\bfs\huge #1}

\usetikzlibrary{%
  arrows.meta,%
  %decorations.pathreplacing,%
  decorations.markings,%
  shapes.misc,%
  patterns
}

%本文
\begin{document}
\frontmatter
% 書く!!
\Chapter{まえがき}

本書は,東京大学理学部数学科の学生有志が製作し,
東京大学で開催されている学園祭の数学科学術企画「ますらぼ」において配布しているものです.
今回初めて手に取っていただいた方もそうでない方も,楽しんでいただけたら幸いです.

読者の皆さんの中で,難しい本を読むときに「まえがき」だけ読んで満足してしまうという方はいらっしゃるでしょうか.
なぜこんなことを聞くかというと,私自身がそうだからです.
とくに数学書を読むときにそれをやると,何だかわかった気になって良い気分になります.
私もそういうまえがきを書いてみたいと思ったのですが,せっかく書いた文章が読まれずに終わってしまうことは避けたいので,
本文の内容に関係のあることは書かないことにします.
従って,本文の内容に興味がある方は,このまえがきを読み飛ばしてください.

さて,これは自分の母親から聞いた話なのですが,私は小学1年生のとき,次の式を「理解できなかった」そうです.
\[ 1+1=2 \]
今となっては当たり前ですし,なぜ「理解できなかった」のかもはや不明ですが,
当時の自分がどうしてわからなかったのかここで考えてみたいと思います.

小学1年生の算数の授業では,初めから$1+1=2$を学ぶわけではありません.
具体的な計算を習う前に,十分慣れ親しんでおかなければならない概念があります.
それは「数の概念」です.
文部科学省が出している小学校学習指導要領によれば,小学1年生の算数の目標の中に次のようなものがあります.
\begin{quote}
具体物を用いた活動などを通して,数についての感覚を豊かにする.
数の意味や表し方について理解できるようにするとともに,加法及び減法の意味について理解し,
それらの計算の仕方を考え,用いることができるようにする.
\end{quote}
「具体物を用いた活動」とは,簡単に言えば個数を比較したり順序を考えたりすることです.
こうした体験を通じて,数の概念の有用性や便利さを覚え,身につけていくわけです.
しかし,個数や順序を数で表すだけでは何も面白くありません.
物事を数を用いて表すもうひとつの理由が,$1+1=2$,すなわち計算です.
どんなものの個数も,数で表すことによって,たし算やひき算ができます.
そしてその計算は,数えるものが何であろうと変わりません.

このように考えてみると,私たちは小学1年生の時点で「抽象化」の大きな一歩を踏み出していることがわかります.
今までは
石ころ1個と石ころ1個で石ころ2個,
車1台と車1台で車2台
など別々に考えていたものを「数」という観点で見直すと,どちらも$1+1=2$と表現できます.
逆に言えば,小学校で学ぶ分には,$1+1=2$にはそのくらいの意味しかないということになりますが,
当時の私はそれ以上の何かを求めていたのかもしれません.
あるいは,それが物事を抽象化して表したものであるということをきちんと理解しきれていなかったのかもしれません.

ところで,中学校に入学すると,今まで算数と呼ばれていたらしいものは数学と言われるようになります.
名前が変わるということは何かが違うはずなのですが,いったい何が違うのでしょうか.
私は次のように考えています.
\begin{quote}
\begin{itemize}
\item[算数:]種々の問題を日常生活の中での感覚も助けにしながら論理的に解決する
\item[数学:]種々の問題を日常生活から離れて論理的に解決する
\end{itemize}
\end{quote}
最近の中学校の教科書は,日常生活の中での感覚もできるだけ助けになるようにかなり工夫して書かれているようですが,
それでも実際には,日常から離れ,言語と論理を駆使して問題を解決します.
2次方程式を解いたり,図形の証明問題を解いたりするときに,日常での経験はほとんどあてになりません.
これは他の4教科---国語,社会,英語,理科---とは明らかに異なる点です.
もちろん,数学はそういうところにも面白さがあるわけですが,
世の中の多くの人が数学に興味がない,あるいは数学が好きではない理由も,
ここにある気がしてならないのは私だけでしょうか.

長々とまとまらない話をしてきましたが,そういうわけで,数学という営みは
積極的に日常から離れることを要求するものなのです.
無論,数学に出てくる様々な概念の中には,日常の中の発想から生まれたものも多くあります.
しかし,それを抽象化し,日常から切り離された論理の世界で発展させていくのが数学です.
たとえアイデアの着想に至るまでに直観の力を借りたとしても,
直観によらない強靭な思考を駆使し,そのアイデアをきちんと数学に昇華させなければ,
せっかくの発想も日の目を見ることは無いでしょう.

...と,ここまで書いてみましたが,数学がどんなものなのか,少しでもわかった気になっていただけたでしょうか.
読む人にもよるかと思いますが,私の試みがうまくいったという方が多くいらっしゃることを期待します.

さて,本書の内容は,いずれも執筆者の個性溢れる「数学」となっています.
必ずしも数学を専門としない方向けに書かれているとはいえ,
もしかすると少しばかり日常を離れなければならないかもしれません.
私たちの数学が,少しでも皆さんの「非日常」に貢献できることを願っています.

最後に,本書の製作にあたって執筆や印刷,製本に協力してくれたすべての方々,
そして本書をお読みくださっているすべての皆さまに感謝いたします.

\begin{flushright}
編者を代表して

2018年5月
\end{flushright}
\clearpage % tocloft パッケージを使う場合は自分で \clearpage しないといけない
\tableofcontents
\mainmatter
{\subfile{ueda}}
{\subfile{hamada}}
% \cftaddtitleline{toc}{chapter}{圏論が分かる4コマ漫画(小林)}{章間}
% このようにすると好きなように目次を変更できる
\backmatter
%\Chapter{編集後記}を入れても良い
% 書きかえる!!
\thispagestyle{empty}
\vspace*{10zw}
\vfill

\parindent=0pt
\begin{picture}(110,1)
\setlength{\unitlength}{1truemm}
\put(5,2){\Large\textbf{$e^{\pi i}sode$ Vol.7 }} 
\thicklines
\put(0,1){\line(2,0){110}}
\thinlines
\put(0,0){\line(2,0){110}}
\end{picture}

\small{2018年5月19日発行}\\
 \normalsize{著 者・・・・・東京大学理学部数学科有志}\\
 \normalsize{発行人・・・・・濱田 昌隆}\\
\begin{picture}(100,1)
\setlength{\unitlength}{1truemm}
\thinlines
\put(0,1){\line(2,0){110}}
\thicklines
\put(0,0){\line(2,0){110}}
\put(0,-5){\small{\copyright  Students at Department of Mathematics,The University of Tokyo 2018 Printed in Japan}}
\end{picture}

\end{document}